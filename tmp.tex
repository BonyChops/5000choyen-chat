\documentclass[a4j, titlepage, dvipdfmx]{jarticle}
\usepackage{float}
\usepackage[dvipdfmx]{graphicx}
%\usepackage{mediabb}
\makeatletter
%https://qiita.com/ta_b0_/items/2619d5927492edbb5b03
\usepackage{listings,jlisting} %日本語のコメントアウトをする場合jlstlistingが必要
%ここからソースコードの表示に関する設定
\lstset{
  basicstyle={\ttfamily},
  identifierstyle={\small},
  commentstyle={\smallitshape},
  keywordstyle={\small\bfseries},
  ndkeywordstyle={\small},
  stringstyle={\small\ttfamily},
  frame={tb},
  breaklines=true,
  columns=[l]{fullflexible},
  numbers=left,
  xrightmargin=0zw,
  xleftmargin=3zw,
  numberstyle={\scriptsize},
  stepnumber=1,
  numbersep=1zw,
  lineskip=-0.5ex
}
%ここまでソースコードの表示に関する設定

\if0
---------------------------------こめこめこめこめこめこめこめこめこめこめこめ
\renewcommand{\thefigure}{\arabic{figure}}
\@addtoreset{figure}{section}
\makeatother
\makeatletter
\renewcommand{\thetable}{\arabic{table}}
\@addtoreset{table}{section}
\makeatother
\makeatletter
\renewcommand{\theequation}{\arabic{equation}}
\@addtoreset{equation}{section}
\makeatother
---------------------------------こめこめこめこめこめこめこめこめこめこめこめ
\fi

\usepackage{pdfpages}
\usepackage{amssymb}
\usepackage{amsmath}
\usepackage{url}
\title{電気回路II}

\author{鈴木颯太\thanks{長野工業高等専門学校 電子情報工学科}}
\date{令和2年5月6日}
\begin{document}
\maketitle
\section*{10.2 【閉路電流法】}
ありがとうございました。 \\
教科書の図より,閉回路$S_a$,$S_b$の式をつくる.
\begin{eqnarray}
    S_a : I_a(Z_1+Z_2)+I_b(Z_2) &=& E\\
    S_a : I_a(40-j10)-j10I_b &=& E
\end{eqnarray}
\begin{eqnarray}
    S_b : I_a Z_2 + (Z_2 Z_3 Z_4) = 0\\
    S_b : -j10 I_a  + (10+j10) = 0
\end{eqnarray}

よって,
\[
  \left[
    \begin{array}{rr}
      40-j10 & -j10  \\
      -j10 & 10+j10
    \end{array}
  \right]
  \left[
    \begin{array}{r}
      I_a  \\
      I_b
    \end{array}
  \right] =
  \left[
    \begin{array}{c}
      400  \\
      0 \\
    \end{array}
  \right]
\]

\begin{eqnarray}
\Delta &=&
\left|
    \begin{array}{rr}
    40-j10 & -j10  \\
    -j10 & 10+j10  \\
  \end{array}
\right| \\
&=& (400+j400-j100+100)-(-100) \\
&=& 600+j300
\end{eqnarray}

\begin{eqnarray}
    I_a &=& \frac{1}{\Delta}
\left|
    \begin{array}{rr}
    100 & -j10  \\
    0 & 10+j10  \\
  \end{array}
\right| \\
&=& \frac{1000+j1000}{600+j300}\\
&=& \frac{10+j10}{6+j3}\\
&=& \frac{90+j30}{45}\\
&=& 2+j \frac{2}{3} \\
&=& \underline{2+0.667[A]}
\end{eqnarray}

\begin{eqnarray}
    I_b &=& \frac{1}{\Delta}
    \left|
    \begin{array}{rr}
        40-j10 & 100  \\
        -j10 & 0  \\
      \end{array}
    \right| \\
    &=& \frac{j1000}{600+j300} \\
    &=& \frac{30+j60}{45} \\
    &=& \frac{2}{3} + j \frac{4}{3} \\
    &=& \underline{0.667+j1.333[A]}
\end{eqnarray}
よって,
\begin{eqnarray}
    I_1 &=& I_a = \underline{2+j0.667[A]} \\
    I_2 &=& I_a+I_b= \underline{2.667+j2[A]} \\
    I_3 &=& -I_b = \underline{-0.667 - j1.333[A]}
\end{eqnarray}
\newpage

\section*{10.4 【閉路電流法】}
教科書の図より,閉回路$S_a$,$S_b$,$S_c$の式をつくる.
\begin{eqnarray}
    S_a : I_a (j6+8)-8I_b=20\\
\end{eqnarray}
\begin{eqnarray}
    S_b : -8 I_a+I_b(8+10+4)+4I_c&=&0\\
    -8 I_a+22I_b+4I_c &=& 0\\
\end{eqnarray}
\begin{eqnarray}
    S_c : -8I_b + (4-3j)I_c &=& j20
\end{eqnarray}

よって,
\[
  \left[
    \begin{array}{rrr}
      j6+8 & -8 & 0  \\
      -8 & 22 & 4  \\
      0 & 4 & 4-3j
    \end{array}
  \right]
  \left[
    \begin{array}{rrr}
      I_a  \\
      I_b \\
      I_c
    \end{array}
  \right] =
  \left[
    \begin{array}{c}
      20  \\
      0 \\
      j20
    \end{array}
  \right]
\]
\begin{equation}
    \Delta = 716 + j96
\end{equation}

\begin{eqnarray}
    I = I_b &=& \frac{1}{\Delta}
\left|
    \begin{array}{ccc}
    j6+8 & 20 & 0 \\
    -8 & 0 & 4  \\
    0 & j20 & 4-3j
  \end{array}
\right| \\
&=& \frac{-(j80(j6+8))+(160(4-3j))}{716+j96}\\
&=& \underline{1.33-j1.79[A]}
\end{eqnarray}
\newpage

\section*{10.5 【オーウェンブリッジ】}
教科書図10.10と照らし合わせて,$Z$の値を決める.
\begin{eqnarray}
    Z_1 &=& R_1 + j\omega L_1\\
    Z_2 &=& R_2 \\
    Z_3 &=& R_3 + \frac{1}{j\omega C_3} \\
    Z_4 &=& \frac{1}{j\omega C_4}
\end{eqnarray}
以下の平衡条件である式を使って求める.
\begin{equation}
    Z_1Z_4 = Z_2Z_3
\end{equation}
\begin{eqnarray}
    \frac{R_1}{j\omega C_4} + \frac{L_1}{C_4} &=& R_2R_3 + \frac{R_2}{j\omega C_3}\\
    \frac{C_3R_1}{C_4}+ j\omega \frac{C_3L_1}{C_4} &=& j\omega C_3 R_2R_3+R_2\\
    C_3 + j\omega C_3 L_1 &=& j\omega C_3 C_4 R_2 R_3 + C_4 R_2 \\
\end{eqnarray}
実部,虚部は等しいものとすると,
\begin{eqnarray}
    R_1 &=& \frac{C_4R_2}{C_3} = \underline{40[\Omega]}\\
    L_1 &=& C_4R_2R_3 = \underline{2.0 \times 10^{-2} [H]}
\end{eqnarray}
\newpage

\section*{10.6 【マクスウェルブリッジ】}
教科書図10.10と照らし合わせて,$Z$の値を決める.
\begin{eqnarray}
    Z_1 &=& R_1 + j\omega L_1\\
    Z_2 &=& R_2 \\
    Z_3 &=& R_3  \\
    Z_4 &=& \frac{1}{\frac{1}{4} + j\omega C_4}\\
    &=& \frac{\frac{R_4}{1+R_4j\omega C_4}}{R_4 + \frac{1}{j\omega C_4}}
\end{eqnarray}
以下の平衡条件である式を使って求める.
\begin{equation}
    Z_1Z_4 = Z_2Z_3
\end{equation}

\begin{eqnarray}
    \frac{\frac{R_4}{j\omega C_4}}{R_4+ \frac{1}{j\omega 4}}(R_1+j\omega L_1) &=& R_2R_3 \\
    R_4(R_1+j\omega L_1) &=& R_2R_3(1+j\omega C_4R_4)
\end{eqnarray}
実部,虚部は等しいものとすると,
\begin{eqnarray}
    R_1 &=& \underline{\frac{R_2R_3}{R_4}}\\
    L_1 &=& \underline{C_4R_2R_3}
\end{eqnarray}
\newpage

\section*{10.7 【ウィーンブリッジ・各種端数の決定】}
教科書図10.10と照らし合わせて,$Z$の値を決める.
\begin{eqnarray}
    Z_1 &=& R_1\\
    Z_2 &=& R_2 \\
    Z_3 &=& R_3 - j\frac{1}{\omega C_3}  \\
    Z_4 &=& \frac{\frac{R_4}{j\omega C_4}}{R_4 + \frac{1}{j\omega C_4}}
\end{eqnarray}

以下の平衡条件である式を使って求める.
\begin{equation}
    Z_1Z_4 = Z_2Z_3
\end{equation}

\begin{eqnarray}
    \frac{\frac{R_1R_4}{j\omega C_4}}{R_4 + \frac{1}{j\omega C_4}} &=& R_2R_3-j\frac{R_2}{\omega C_3} \\
    j\omega R_1R_4C_3 &=& -\omega^2 C_3 C_4 R_2 R_3 - R_3 R_4 \times j \omega C_4 + 1 \\
\end{eqnarray}

実部,虚部は等しいものとすると,
\begin{eqnarray}
    \omega &=& \underline{\sqrt{1/(C_3 C_4 R_3 R_4)}}\\
    R_1 &=& \underline{(C_3R_2R_3+C_4R_2R_4)/(C_3R_4)}
\end{eqnarray}

\section{レポートの下書き}
以下レポートの手書きでの下書きである.(次ページから)
\end{document}